\documentclass[../../../main.tex]{subfiles}


\begin{document}







\section{Elements enters i extensions enteres} 

\begin{defi}[Enter algebraic]\label{def:enter}\label{def:enterSobreA}\index{Enter sobre $A$}
Sigui $A$ un anell commutatiu i $A\subseteq B$ una extensió d'anells. Es diu que $b\in B$ és \textit{enter sobre $A$} si és una arrel d'un polinomi mònic amb coeficients en $A$, és a dir, d'una equació com
\begin{equation}
    \notag
    X^n+a_1X^{n-1}+\cdots+a_{n-1}X+a_n
\end{equation}
\end{defi}


\begin{prop}
\label{prop:enterSobreA} Sigui $A\subseteq B$ una extensió d'anells i $b\in B$, aleshores són equivalents
\begin{enumerate}[(i)]
    \item $b$ és enter sobre $A$.
    \item $A[b] = \left\{\sum_{i=0}^n a_ib^{i},\;a_i\in A,\;n\in\mathbb{N}\right\}\subseteq B$ és un $A$-mòdul finitament generat.
    \item $A[b]\subseteq C$ on $C\subseteq B$ és un subanell tal que és $A$-mòdul finitament generat.
    \item Existeix un $A[b]$-mòdul fidel $M$ tal que $M$ és $A$-mòdul finitament generat.
\end{enumerate}
\end{prop}
\begin{proof}
La implicació de (i) a (ii) és força senzilla, ja que tenim $b^n = \sum_{i=1}^na_ib^{n-i}$ i aleshores $A[b]\subseteq \bigoplus_{j=1}^{n-1}Ab^j$ i per tant, $A[b]$ és un $A$-mòdul finitament generat.

La implicació de (ii) a (iii) és directa ja que podem agafar $C = A[b]$ que és un anell. La implicació (iii) a (iv) també és ràpida perquè podem observar que $C$ és un $A[b]$-mòdul fidel ja que si $yC = 0$ aleshores $y1 = 0$ cosa que implica $y = 0$.

Provem doncs la implicació de (iv) a (i). Considerem l'aplicació $f:M\to M$ que assigna $f(m) = mb$. Aleshores és $A$-lineal i $M$ és un $A$-mòdul finitament generat i $f(M)= bM\subseteq M$, per tant ara podem aplicar el truc determinantal \ref{prop:trucDeterminantal} i per tant existeixen $a_0,a_1,\ldots,a_{n-1}\in A$ amb $a_0 = 1$ complint
\begin{equation}
    \notag
    \left(\sum_{i=1}^na_{n-i}b^{i}\right)m = 0
\end{equation}
per tot $m\in M$ i com que $M$ és fidel, aleshores $\sum_{i=1}^n a_{n-i}b^{i} = 0$ i per tant $b$ és enter sobre $A$.
\end{proof}


\begin{coro}
\label{coro:entersSobreA} Siguin $b_1,\ldots,b_s\in B$ enters sobre $A$. Aleshores $A[b_1,\ldots,b_s]\subseteq A$ és un $A$-mòdul finitament generat.
\end{coro}


\begin{coro}
\label{coro:clausuraAEnB}\index{Clausura entera de $A$ en $B$} Definim $\overline{A}^B:=\{b\in B\;:\;b$ és enter sobre $A\}$ com la \textit{clausura de $A$ en $B$}. Aleshores $\overline{A}^B$ és un subanell de $B$.
\end{coro}
\begin{proof}
Hem de provar que si tenim $b_1,b_2\in \overline{A}^B$ aleshores $b_1+b_2$ i $b_1b_2$ també són enters sobre $A$. Com que $A[b_1,b_2]$ és un $A$-mòdul finitament generat tenim
\begin{equation}
    \notag
    \left.
    \begin{array}{rl}
        A\subseteq A[b_1+b_2]\subseteq A[b_1,b_2] &  \\
        A\subseteq A[b_1b_2]\subseteq A[b_1,b_2] & 
    \end{array}
    \right\} \Longrightarrow b_1+b_2,b_1b_2\in \overline{A}^B
\end{equation}
per la proposició \ref{prop:enterSobreA}. 
\end{proof}

\begin{nota}
Si $\overline{A}^B = B$ diem que $B$ és  enter sobre $A$ o bé que l'extensió és entera;\label{def:integramentTancat} i si $A$ és domini d'integritat i $B$ el seu cos de fraccions, aleshores $\overline{A}^B$ se li'n diu \textit{clausura entera}. Si $\overline{A}^B = A$ diem que és \textit{íntegrament tancat}\index{Íntegrament tancat}.
\end{nota}


\begin{defi}
[Íntegrament tancat]\label{def:integramentTancat2}\index{Íntegrament tancat} Direm que un domini d'integritat $A$ és \textit{íntegrament tancat} si $A$ és íntegrament tancat en el seu cos de fraccions.
\end{defi}

\begin{ej}
Si $A$ és un anell factorial aleshores $A$ és íntegrament tancat. 

Sigui $\frac{a}{s}\in K$, on $K$ és el cos de fraccions, tal que
\begin{equation}
    \notag
    \left(\frac{a}{s}\right)^n+\left(\frac{a}{s}\right)^{n-1}a_1+\cdots+\left(\frac{a}{s}\right)a_{n-1}+a_n = 0
\end{equation}
amb $a_i\in A$ per tot $i$. Podem suposar que $\gcd(a,s) = 1$ i aleshores
\begin{equation}
    \notag
    a^n+sa^{n-1}+a_1+\cdots+s^{n-1}aa_{n-1}+s^na_n=0
\end{equation}
obtenim $a^n = s(-a^{n-1}a_1-\cdots-s^{n-1}a_n)$ per tant teim que $s$ divideix a $a^n$ però $\gcd(s,a) = 1$, per tant $s\in A^*$ i obtenim que $\frac{a}{s}\in A$.

En particular, si $A$ és un domini d'ideals principals, aleshores $A$ és íntegrament tancat.
\end{ej}

\begin{coro}
\label{coro:extensioEnteraTriple} Siguin $A\subseteq B\subseteq C$ anells tals que $A\subseteq B$ i $B\subseteq C$ són enters aleshores $A\subseteq C$ és una extensió entera.
\end{coro}
\begin{proof}
Si
$x\in C$,
aleshores tenim l'equació 
$x^n+b_1x^{n-1}+\cdots+b_n = 0$.
Com que l'anell 
$B'=A[b_1,\ldots,b_n]$
és finitament generat com a 
$A$-mòdul
i $B'[x]$ és finitament generat com a 
$B'$-mòdul
tenim que $B'[x]$
és finitament generat com a $A$-mòdul,
que per la proposició \ref{prop:enterSobreA} aleshores $x$ és enter sobre $A$.
\end{proof}

\begin{ej}
Sigui $A = k[t^2,t^3]\subseteq k[t]$. Sigui $K$ el cos de fraccions de $A$. Aleshores $k[t]\subseteq K\subseteq k(t)$ però tot és igualtat ja que $t^3/t\in K$. Ara bé, $t\not\in A$.
\end{ej}

Reformulo el teorema de Krull-Akizuki amb aquests termes.

\begin{ter}
[Krull-Akizuki]\index{Teorema de Krull-Akizuki} Sigui $A$ un domini d'integritat, noetherià i de dimensió 1 ($\p\in\spec{A},\p\neq 0$ aleshores $\p$ maximal). Aleshores, sigui $K$ el cos de fraccions, $\overline{A}^K$ és domini d'integritat, noetherià i de dimensió 1. Més en general, si tenim una extensió finita $K\subseteq L$ de cossos, aleshores $\overline{A}^L$ satisfà el mateix.
\end{ter}

Només demostrarem el tema de la $L$, ja que l'altre tema ja ho hem demostrat abans no sé per què.

\begin{ter}
Suposem que $A$ és domini d'integritat tancat. Suposem que $K$ és el cos de fraccions de $A$. Sigui $L\subseteq K$ una extensió finita de cossos separable. Sigui $B = \overline{A}^L$. Aleshores, existeix una base $\{v_1,\ldots,v_n\}$ de $L$ sobre $K$ tal que $B\subseteq \bigoplus_{i=1}^nAv_i$.
\end{ter}
\begin{proof}
Sigui $v\in L$ algebraic sobre $K$, aleshores existeix $a_rv^r+a_{r-1}v^{r-1}+\cdots+a_1v+a_0 = 0$ amb $a_i\in A$ i ara multiplicant-ho tot per $a_rb_{r-1}$ o no sé, obtenim
\begin{equation}
    \notag
    (a_rv)^r+b_{r-1}(a_rv)^{r-1}+\cdots+a_0 = 0
\end{equation}
per tant $a_rv$ enter sobre $A$, $a_r\in A\subseteq K$. Aleshores existeix base de $L$ sobre $K$ amb elements enters. Diguem-li $\{u_1,\ldots,u_n\}$ a aquesta base. Llavors, existeix una base dual $\{v_1,\ldots,v_n\}$ de foram que $\forall x\in B$ tenim $x = \sum_{j=1}^nx_jv_j$ amb $x_j = \mathrm{tr}(u_jx_j)$ per tota $j$. Ara si $x\in B$ també $xu_j\in B$ i així $\mathrm{tr}(xu_j)\in A$ ja que el polinomi mínim d'un element enter té els seus coeficients en $A$. Per tant $x_j\in A$, $\forall j$ i així obtenim el que volíem.
\end{proof}

Observem el que acabo de dir aquí al final. Si $a\in\overline{A}^L$ i $p(x)$ és el polinomi mínim de $a$ en $K[X]$ i $q(x)$ és un polinomi mònic en $A[X]$ que té $a$ com a arrel, aleshores el fet que $q(a) = 0$ implica que $p\mid q$ per ser $p$ el mínim. Però aleshores $p(b) = 0$ implica $q(b) = 0$ per qualsevol altre $b$ conjugat de $a$ i així $p(b) = 0$ implica que $b$ e´s enter sobre $A$ (i.e., $b\in\overline{A}^L$) i així $p(x)\in \overline{A}^K[X] = A[X]$ perquè $A$ és íntegrament tancat.


\begin{coro}
Amb les mateixes condicions que el teorema, si $A$ és noetherià, $\overline{A}^L$ és noetherià. També si $A$ és domini d'ideals principals, aleshores $\overline{A}^L$ és un $A$-mòdul lliure.
\end{coro}
\begin{proof}
Simplement s'ha de constatar que $\overline{A}^L\subseteq \bigoplus_{j=1}^n Av_j\subseteq L$ pel lema i $Av_j$ és noetherià per tota $j$, per tant la suma directa també és noetherià. 
\end{proof}

\begin{lema}
$A\subseteq B$ domini d'integritat, tal que $\dim A = 1$. Si $A\subseteq B$ és una extensió d'anells finita, aleshores $\dim B = 1$.
\end{lema}
\begin{proof}
Sigui $\mathfrak{q}\in\spec{B}$, $\mathfrak{q}\neq 0$, aleshores veurem que $\mathfrak{q}$ és maximal.

$\p=\mathfrak{q}^c\in\spec{A}$, $\p\not=0$ en efecte $b\in\mathfrak{q}$, $b\neq0$ $\exists b^n+a_1b^{n-1}+\cdots a_n = 0$, $a_i\in A$, $\forall i$. Prenem la de grau mínim. Podem suposar $a_n\neq 0$. Llavors, $a_n = -b^n-a_1b^{n-1}-\cdots-a_{n-1}b\in\mathfrak{q}\cap A = \p$. Tenim la inclusió $A/\p\hookrightarrow B/\mathfrak{q}$ on el primer és cos i el segon és domini. Aquesta és una extensió finita perquè per hipòtesis $A\hookrightarrow B$ és finita. Per tant, tenim un cos, adjunto un nombre finit d'elements, i tinc una extensió finita? Ha de ser que $B/\mathfrak{q}$ sigui cos i per tant $\mathfrak{q}$ és maximal com volíem.
\end{proof}

\begin{coro}
Si $A$ és domini d'ideals principals i $K$ el seu cos de fraccions, i considerem $A\subseteq L$ una extensió algebraica finita aleshores
\begin{enumerate}[(1)]
    \item $\overline{A}^L$ domini noetherià de dimensió 1.
    \item $\overline{A}^L$ $A$-mòdul lliure de rang finit
\end{enumerate}
\end{coro}














































\end{document}