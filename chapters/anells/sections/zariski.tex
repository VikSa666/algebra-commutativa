\documentclass[../../../main.tex]{subfiles}



\begin{document}




\section{Topologia de Zariski}


Recordem que hem definit $\spec{A} = \{\mathfrak{p}\subseteq A\;:\;\mathfrak{p} \text{ primer }\}$. És un PoSet\footnote{Partially ordered set, un conjunt parcialment ordenat} per les inclusions. Clarament $\max\spec{A}$ és el conjunt de maximals de $A$ i el mateix amb els mínims. Existeix una correspondència bijectiva entre l'espectre de $A/I$ i el conjunt d'ideals primers d'$A$ que contenen $I$:
\begin{equation}
    \label{eq:correspondenciaBijectivaSpec}
    \{\mathfrak{p}\in\spec{A/I}\}\longleftrightarrow \{\mathfrak{p}\in\spec{A}\;:\;I\subseteq\mathfrak{p}\}
\end{equation}

A continuació introduirem la topologia de Zariski en $\spec{A}$. El que passa és que els ideals de $\spec{A}$ s'organitzen en cadenes, per estar parcialment ordenat. Aleshores volem estudiar aquestes cadenes de forma més enllà que conjuntística. 

\begin{defi}
[Varietat algebraica afí]\label{def:varietatAlgebraicaAfi} Si prenem $k[X_1,\ldots,X_n]$, on $k$ és un cos, l'anell de polinomis sobre $k$, i prenem un ideal $I$ d'aquest anell, llavors es defineix $\mathbb{V}(I)$ com el conjunt de punts de $k^n$ que anul·len tot polinomi $f\in I$. Se'n diu \textit{varietat algebraica afí}\index{Varietat algebraica afí}.
\end{defi}

Llavors, considerem $\{f_1,\ldots,f_n\}\subseteq k[X_1,\ldots,X_n]$ i aleshores el conjunt $\{p\in k^n\;:\;f_i(p) = 0,\;\forall i\}$ és un conjunt també de l'espai afí. Si $k = \overline{k}$, aleshores existeix una bijecció entre els conjunts aquests i els ideals $I$ que són radicals. Tot això està als apunts d'``Anells de Polinomis en Diverses Variables''.

Aleshores, la idea de Zariski va ser introduir una topologia en la família de varietats afins. La topologia que descriu és la que ve definida pels tancats que són les varietats afins.

Veiem una sèrie de propietats d'aquestes varietats. És clar que si $E\subseteq F$, aleshores $\mathbb{V}(F) = \mathbb{V}(E)$.

\begin{prop}
Vegem algunes propietats.
\begin{enumerate}[(i)]
    \item Si $I = \langle E\rangle$, aleshores $\mathbb{V}(E) = \mathbb{V}(I) = \mathbb{V}(r(I))$.
    \item $\mathbb{V}(\emptyset) = \spec{A}$, $\mathbb{V}(A) = \emptyset$.
    \item $\{E_i\}_{i\in I}$, llavors
    \begin{equation}
        \notag
        \mathbb{V}\left(\bigcup_{i\in I}E_i\right) = \bigcap_{i\in I}\mathbb{V}(E_i)
    \end{equation}
    \item $\mathbb{V}(I\cap J) = \mathbb{V}(I)\cup\mathbb{V}(J) = \mathbb{V}(IJ)$, on $I$ i $J$ són ideals.
\end{enumerate}
\end{prop}
\begin{proof}
\begin{enumerate}[(i)]
    \item $E\subseteq I$ implica $\mathbb{V}(I)\subseteq \mathbb{V}(E)$. $\mathfrak{p}\supseteq E\Rightarrow I\subseteq \mathfrak{p}\Rightarrow \mathfrak{p}\subseteq\mathbb{V}(I)\Rightarrow \mathbb{V}(E)\subseteq \mathbb{V}(I)$
    $I\subseteq \mathbb{V}(I)\Rightarrow \mathbb{V}(r(I))\subseteq \mathbb{V}(I)$. $\mathfrak{p}\supseteq I\Rightarrow r(I)\Rightarrow \mathfrak{p}$
    \item Ok
    \item $\mathbb{V}(\bigcup_{i\in I} E_i)=\bigcap_{i\in I}\mathbb{V}(E_i)$. $E_i\subseteq\bigcup_i E_i\Rightarrow \mathbb{V}(\bigcup_i E_i)\subseteq \mathbb{V}(E_i)$ per tota $i$, y llavors $\mathbb{V}(\bigcup_i E_i)\subseteq\bigcap_i\mathbb{V}(E_i)$.
    \item $I\cap J\subseteq I$ y $I\cap J\subseteq J$ implica $\mathbb{V}(I\cap J)\supseteq \mathbb{V}(I),\mathbb{V}(J)$ que implica $\mathbb{V}(I)\cup\mathbb{V}(J)\subseteq\mathbb{V}(I\cap J)$. D'altra banda, $IJ\subseteq I\cap J$ implica $\mathbb{V}(I\cap J)\subseteq \mathbb{V}(IJ)$.
    
    Ara si $\mathfrak{p}\supseteq IJ$, llavors $I\subseteq \mathfrak{p}$ o bé $J\subseteq \mathfrak{p}$ por el lema de contenció de primers \ref{lema:evitacioPrimersInterseccio}, i.e. $\mathfrak{p}\in \mathbb{V}(I)$ o bé $\mathfrak{p}\in \mathbb{V}(J)$ és a dir que pertanyen a la unió, com volíem.
\end{enumerate}
\end{proof}

\begin{defi}
[Topologia de Zariski]\label{def:topologiazariski}\index{Topologia de Zariski} Definim la \textit{topologia de Zariski} en $\spec{A}$ com la topologia en la qual els tancats són de la forma
\begin{equation}
    \notag
    \mathbb{V}(E),\qquad \forall E\subseteq A
\end{equation}
i equivalentment $\mathbb{V}(I)$ per a $I$ ideal d'$A$ tal que és radical.
\end{defi}

Ara hem de veure que això és una topologia, però podem veure les propietats de la definició utilitzant la proposició que tot just hem demostrat.

Com a observació, els tancats, és a dir, $\mathbb{V}(E) = \mathbb{V}(\langle E\rangle)$ s'equivalen amb $\spec{A/I}$ i per tant aquests també són tancats de la topologia de Zariski. És a dir, existeix un homeomorfisme entre l'espai topològic de Zariski y l'espai topològic que té per tancats a $\spec{A/I}$, per a qualsevol $I$ ideal.

Els oberts d'aquesta topologia de Zariski són molt grans, en canvi els tancats molt petits. Per exemple, si tinc $\mathfrak{m}\in\max A$, és a dir, un ideal maximal,, aleshores $\spec{A}\setminus\{\mathfrak{m}\}$ és un obert. El mateix $\mathfrak{m}$ és un tancat, perquè és $\{\mathfrak{m}\} = \mathbb{V}(\mathfrak{m})$.

Aleshores, el que farem a continuació és trobar una base per poder caracteritzar millor aquesta topologia, que és una mica rara. Per tot $a\in A$, denotarem $X = \spec{A}$ i $X_a = \{\mathfrak{p}\in\spec{A}\;:\;a\not\in\mathfrak{p}\}$ que es pot escriure com $X_a = \spec{A}\setminus\mathbb{V}(A)$. Aquest $X_a$ és un obert i aleshores veiem que $\{X_a\}_{a\in A}$ és una base d'oberts. A vegades $X_a$ s'escriu $D(a)$. Si $\mathcal{M} = \spec{A}\setminus\mathbb{V}(E) = \bigcup_{a\in E}X_a$. Aquests $X_a$ són els \textit{oberts bàsic}\index{Oberts bàsics}.

\begin{prop}
\label{prop:obertsbasics} \begin{enumerate}[(1)]
    \item $X_a\cap X_b = X_{ab}$.
    \item $X_a = \emptyset$ si, i només si, $a$ és nilpotent.
    \item $X_a = \spec{A}$ si, i només si, $a$ és unitat, i.e. $a\in A^*$.
    \item $X_a = X_b$ si, i només si, $\mathrm{rad}((a)) = \mathrm{rad}((b))$.
    \item $X$ és quasi-compacte.
    \item $X_a$ són quasi-compactes per tot $a$.
    \item Si $\mathcal{U}\subseteq X$ obert, aleshores és quasi-compacte si, i només si $\mathcal{U}$ és unió finita d'oberts bàsics.
\end{enumerate}
\end{prop}
\begin{proof}
\begin{enumerate}[(1)]
    \item $a,b\not\in\mathfrak{p}$ si, i només si, $ab\not\in\mathfrak{p}$, per la definició d'ideal primer.
    \item $a$ és nilpotent si, i només si,  $a\in\eta(A) = \bigcap_{\mathfrak{p}\in\spec{A}}\mathfrak{p}$.
    \item Trivial.
    \item $X_a\subseteq X_b$ si i només si per tot ideal primer $\mathfrak{p}$, si $a\not\in\mathfrak{p}$ implica $b\not\in\mathfrak{p}$ i això és si i només si per tot primer $\mathfrak{p}$, si $b\in\mathfrak{p}$ aleshores $a\in\mathfrak{p}$ cosa que implica $a\in\mathrm{rad}((b))$. Això és equivalent a veure que $\mathrm{rad}((a))\subseteq \mathrm{rad}((b))$. La inclusió contrària és anàloga.
    \item Si $X = \bigcup_{i\in I} X_{a_i}$ aleshores prenem l'ideal generat per tots els $a_i$, és a dir, $I = (a_i)_{i\in I}$. Llavors,
    \begin{equation}
        \notag
        \mathbb{V}(I) = \bigcap_{i\in I} \mathbb{V}((a_i)) = \spec{A}\setminus\left(\bigcup_{i\in I}(\spec{A}\setminus\mathbb{V}((a_i)))\right)
    \end{equation}
    cosa que implica que $I = A$, ergo $1 = \sum_{j=1}^r \lambda_{i_j} a_{i_j}$ és a dir $\spec{A} = \bigcup_{j=1}^r X_{a_{i_j}}$.
\end{enumerate}
\end{proof}


\begin{defi}
[Aplicació entre espectres]\label{def:aplicacioEspectres}\index{$f^*$}\index{Aplicació entre espectres} Suposem que tenim $f:A\to B$ un morfisme d'anells. Aleshores definim $f^*:\spec{A}\to\spec{B}$ com l'aplicació $f^*(\mathfrak{q}) = f^{-1}(\mathfrak{q}) = \mathfrak{q}^c$.
\end{defi}


\begin{prop}
\label{prop:propietatsAplicacioEspectre} Sigui $f:A\to B$ un morfisme d'anells i $f^*$ l'aplicació de la definició anterior. Aleshores es compleix que
\begin{enumerate}[(i)]
    \item Per a tot $a\in A$, $(f^*)^{-1}(X_a) = X_{f(a)}$. En particular, $f^*$ és contínua.
    \item Per a tot $I\subseteq A$ ideal $(f^*)^{-1}(\mathbb{V}(I)) = \mathbb{V}(I^{e})$ on $I^{e} = \langle f(I)\rangle$.
    \item Per tot $J\subseteq B$ ideal $\overline{f^*(\mathbb{V}(J))} = \mathbb{V}(J^c)$.
    \item Sigui $g:B\to C$ un morfisme d'anells aleshores $(g\circ f)^* = f^*\circ g^*$.
    \item Si $f$ és un epimorfisme aleshores $f^*$ és un homeomorfisme de $\spec{B}$ en $\mathbb{V}(\ker f)$.
    \item $f^*(\spec{B})$ és dens en $\spec{A}$ si i només si $\ker f\subseteq \eta(A)$, en particular si $f$ és monomorfisme, aleshores $f^*(\spec{B})$ és dens en $\spec{ A}$.
\end{enumerate}
\end{prop}
\begin{proof}
\begin{enumerate}
    \item El que farem és calcular directament $(f^*)^{-1}(X_a)$ per tot $a\in A$.
    \begin{equation}
        \notag
        \begin{array}{rl}
            (f^*)^{-1}(X_a) & = \{\mathfrak{q}\in\spec{B}\;:\;f^*(\mathfrak{q})\in X_a\} = \{\mathfrak{q}\in\spec{B}\;:\;f^{-1}(\mathfrak{q})\in X_a\} =  \\
             & = \{\mathfrak{q}\in\spec{B}\;:\;a\not\in f^{-1}(\mathfrak{q})\} = \{\mathfrak{q}\in\spec{B}\;:\;f(a)\not\in\mathfrak{q}\} = X_{f(a)}
        \end{array}
    \end{equation}
    La penúltima igualtat és certa ja que $f(a)\not\in \mathfrak{q}$ si i només si $a\not\in f^{-1}(\mathfrak{q})$.
    
    \item En aquest apartat farem el mateix, calcularem $(f^*)^{-1}(\mathbb{V}(I))$ per tot $I\subseteq A$ ideal.
    \begin{equation}
        \notag\begin{array}{rl}
            (f^*)^{-1}(\mathbb{V}(I)) &  = \{\mathfrak{q}\in\spec{B}\;:\;f^*(\mathfrak{q})\in\mathbb{V}(I)\} = \{\mathfrak{q}\in\spec{B}\;:\;f^{-1}(\mathfrak{q})\in\mathbb{V}(I)\} = \\
             &  = \{\mathfrak{q}\in\spec{B}\;:\;I\subseteq f^{—1}(\mathfrak{q})\} = \{\mathfrak{q}\in\spec{B}\;:\;f(I)\subseteq \mathfrak{q}\} = \\
             & = \mathbb{V}(f(I)) = \mathbb{V}(I^{e})
        \end{array}
    \end{equation}
    
    \item Aquest apartat el demostrarem per inclusions. Primer veiem la inclusió d'esquerra a dreta. Sigui $J\subseteq B$ un ideal qualsevol, aleshores
    \begin{equation}
        \notag
        \begin{array}{rl}
            f^*(\mathbb{V}(J)) & = f^*(\{\mathfrak{q}\in\spec{B}\;:\;J\subseteq\mathfrak{q}\}) = \{f^*(\mathfrak{q})\;:\;J\subseteq \mathfrak{q}\} =  \\
             & = \{f^{-1}(\mathfrak{q})\;:\;J\subseteq\mathfrak{q}\subseteq\{f^{-1}(\mathfrak{q})\;:\;f^{-1}(J)\subseteq f^{-1}(\mathfrak{q})\}\subseteq \\
             & = \subseteq \{\mathfrak{p}\in\spec{A}\;:\;J^c\subseteq\mathfrak{p}\} = \mathbb{V}(J^c)
        \end{array} 
    \end{equation}
    Per tant $f^*(\mathbb{V}(J))\subseteq \mathbb{V}(J^c)$ i per tant $\overline{f^*(\mathbb{V}(J))}\subseteq\mathbb{V}(J^c)$.
    
    Anem a eure l'altra inclusó, primer sabem que $\overline{f^*(\mathbb{V}(J))}\bigcap_{f^*(\mathbb{V}(J))\subseteq\mathbb{V}(L)}\mathbb{V}(L)$ per tant hem de provar que $\mathbb{V}(J^c)\subseteq \bigcap_{f^*(\mathbb{V}(J))\subseteq\mathbb{V}(L)}\mathbb{V}(L)$, és a dir que per tot $\mathbb{V}(L)$ tal que $f^*(\mathbb{V}(J))\subseteq \mathbb{V}(L)$ hem de veure que $\mathbb{V}(J*c)\subseteq\mathbb{V}(L)$. Primer de tot notem que 
    \begin{equation}
        \notag
        f^*(\mathbb{V}(J))\subseteq \mathbb{V}(L)\Longleftrightarrow f^{-1}(\mathbb{V}(J))\subseteq \mathbb{V}(L)\Longleftrightarrow \{f^{-1}(\mathfrak{p})\;:\;J\subseteq \mathfrak{p}\}\subseteq\mathbb{V}(J)
    \end{equation}
    Per tant, $f^*(\mathbb{V}(J))\subseteq\mathbb{V}(L)$ és equivalent a que si $J\subseteq \mathfrak{p}$ on $\mathfrak{p}$ és primer, aleshores $L\subseteq f^{-1}(\mathfrak{p})$, que això és equivalent a si $J\subseteq\mathfrak{p}$ aleshores $f(L)\subseteq\mathfrak{p}$ per tant obtenim que si $f^*(\mathbb{V}(J))\subseteq \mathbb{V}(L)$ aleshores $\mathbb{V}(J)\subseteq \mathbb{V}(f(L))$ i aplicant la proposició anterior i la observació podem obtenir que $f(L)\subseteq J\Longleftrightarrow L\subseteq f^{-1}(J) = J^c$ això implica que $\mathbb{V}(J^c)\subseteq\mathbb{V}(L)$, en resum hem vist que
    \begin{equation}
        \notag
        f^*(\mathbb{V}(J))\subseteq \mathbb{V}(L)\Rightarrow \mathbb{V}(J)\subseteq \mathbb{V}(f(L))\Rightarrow f(L)\subseteq\rad(J)\Rightarrow
    \end{equation}
    \begin{equation}
        \notag
        \Rightarrow L\subseteq \rad(J)^c = \rad(J^c)\Rightarrow \mathbb{V}(J^c) = \mathbb{V}(\rad(J^c))\subseteq\mathbb{V}(L)
    \end{equation}
    Per tant $\mathbb{V}(J^c)\subseteq\overline{f^*(\mathbb{V}(J))}$.
    
    \item És força immediat ja que $(g\circ f)^* = (g\circ f)^{-1} = f^{-1}\circ g^{-1} = f^*\circ g^*$.
\end{enumerate}
\end{proof}

\begin{nota}
Sigui $A$ domini d'integritat amb un únic ideal primer $\mathfrak{p}$ diferent de $0$. Aleshores, si $B = (A/\mathfrak{p})\times K$, on $K = \mathrm{Frac}(A)$ (el cos de fraccions de $A$) i sigui $f:A\to B$ tal que $f(a) = (\overline{a},a/1)$, aleshores $f^*$ és bijectiva però no és homeomorfisme.
\end{nota}

\begin{nota}
Tenim un functor contravariant que va de la categoria dels anells a la categoria d'espais topològics i envia cada anell al seu espectre i cada morfisme $f$ a $f^*$.
\end{nota}








\end{document}